\documentclass[10pt,a4paper]{letter}
\usepackage[latin1]{inputenc}
\usepackage{utopia}
\usepackage{amsmath}
\usepackage{amsfonts}
\usepackage{amssymb}
\usepackage{graphicx}
\usepackage{float}
\usepackage{url}
\usepackage{hyperref}
\usepackage[left=1in, right=1in, bottom=1in, top=1in]{geometry}
\renewcommand{\baselinestretch}{1.2}

\usepackage[table,xcdraw]{xcolor}
\usepackage[dvips]{xcolor}
\definecolorseries


\hypersetup{colorlinks,linkcolor={[cmyk]0,0.88502,1,0},citecolor={[cmyk]0,0.88502,1,0}, filecolor={violet}, urlcolor={[cmyk]0,0.88502,1,0}}  

\begin{document}
{\Large 
	\textbf{Kingdom's Agro-Processing Potential}
} 

The Phnom Penh Post, June 8, 2020, 21:36 ICT \\
By Nith Kosal

Agro-processing is the future of Cambodia's agricultural industry. In recent months, \href{https://www.khmertimeskh.com/50686215/demand-to-modernise-agriculture}{Prime Minister Hun Sen} has called on local and foreign investors to inject more funding into the sector, and \href{https://www.worldbank.org/en/country/cambodia/publication/cambodian-agriculture-in-transition-opportunities-and-risks}{researchers} from institutions like the World Bank agree. Increasing Cambodia's agro-processing capacity has the potential to enhance agricultural and socioeconomic development significantly.

But at the moment, Cambodia is wasting this potential. 

\href{https://www.agriculturenigeria.com/processing/agro-processing/}{Agro-processing}, which links the agricultural and manufacturing sectors, takes raw and intermediate materials --- including crops, livestock, fish and forest materials --- and turns them into finished, marketable products.

For example, Cambodia exports rice to the international market. But other, potentially even more profitable, opportunities exist. Cambodia could add value by processing its rice into other products like sake, vinegar, noodles, bread, or milk.

The benefits of linking agriculture with manufacturing is clear. A \href{https://www.researchgate.net/publication/323749975_Structural_transformation_in_agriculture_and_agro-processing_value_chains}{processing} plant located in a rural area, for example, could serve as an additional buyer for local farmers and potentially help to spur on \href{http://acetforafrica.org/acet/wp-content/uploads/publications/2017/10/ATR17-overview-english-for-web-0911.pdf}{agricultural productivity}. Such a plant would add value and often quality to local domestic products. And, crucially, other economic activities associated with agro-processing, like packaging, retailing, marketing, and transporting goods, all generate jobs.

The experiences of neighboring countries provide the best argument for the kind of growth the agro-processing industry could create in Cambodia. Thailand's agro-processing industry makes use of \href{http://www.boi.go.th/upload/content/Food%20industry_5aa7b40bd758b.pdf}{80 percent} of Thailand's agricultural raw materials. In 2017, that sector was worth US\$102 billion or 23 percent of GDP. Vietnam's agro-processing sector was worth \href{https://www.fas.usda.gov/data/vietnam-food-processing-ingredients-2}{US\$44 billion} or 15 percent of the GDP in 2018. In Thailand, as many as 9,000 agro-processing companies employ more than 1.1 million people, and in Vietnam 9,428 processing companies employ more than 600,000 people.

\textbf{Growth Opportunities}

In comparison, the agro-processing industry contributed only US\$589.83 million or 2.4 percent of Cambodia's GDP in 2018, according to the \href{https://www.nis.gov.kh/nis/NA/NA2018_Tab_files/TAB1-2.htm}{National Institute of Statistics}.

Currently, much of Cambodia's agricultural exports are \href{https://www.netherlandsworldwide.nl/binaries/en-nederlandwereldwijd/documents/publications/2018/10/04/agriculture-in-cambodia/Agriculture+in+Cambodia.pdf}{raw products} like cashew nuts, mangoes, rubber, and cassava. Rather than investing in a domestic processing industry these goods are sent to be processed in Vietnam and Thailand. \href{http://www.mih.gov.kh/File/UploadedFiles/12_9_2016_4_29_43.pdf}{Just 10 percent} of Cambodia's agricultural materials are processed domestically --- a figure that has remained relatively stagnant since 1998.

This figure shows there are obvious investment opportunities and room for gains in the agro-processing industry in Cambodia. And the Cambodian government is making some efforts to help this nascent sector along. 

Under the \href{http://www.mih.gov.kh/File/UploadedFiles/12_9_2016_4_29_43.pdf}{Industrial Development Policy}, which is set to guide decision-making from 2015-2025, the Kingdom plans to increase the export of processed agricultural products to 12 percent by 2025. To achieve this, the government has set meaningful goals, including identifying growth opportunities for Cambodian agro-processing businesses, and identifying priority products for processing and export.  

To attract foreign investment in Cambodian agro-processing, companies registered as Qualified Investment Project (QIP), \href{https://southeastasiaglobe.com/5-reasons-cambodia-is-best-in-the-region-for-foreign-investments/}{can benefit from a tax exemption} of 40 percent and a tax holiday for three years as well as another priority period of three years. QIPs also benefit from Special Economic Zones --- these zones have privileges such as an additional budget for infrastructure, public works, and civil servants who provide immediate assistance as required. Investors can also benefit from free trade agreements such as the EBA, the GSP, and the ASEAN Free Trade Area to export commodities. 

The Kingdom is also working to reduce the cost of electricity, which constitutes a large constraint for processing businesses. In January, Cambodia's Electricity Authority announced that there would be a reduction in electricity tariffs for consumers in six provinces that export to Vietnam in 2020. For agricultural SMEs operating in these provinces, the cost reduced to US\$0.1370 per kWh and US\$ 0.1580 per kWh for large enterprises.

\textbf{Remaining Constraints}

But Cambodia is not yet ready to truly take advantage of agro-processing. According to a \href{http://www.ukabc.org.uk/wp-content/uploads/2017/04/AgriProject_Reporting_FINAL-VERSION-copy.pdf}{2016 survey} from Cambodian market research firm BDLINK, there are many constraints for this sector, such as irregular and insufficient supply of raw materials, a lack of skilled workers for necessary maintenance and operation of the processing machinery as well as unreliable transportation and poor road quality.

Historically, transportation cost has been another big constraint. In 2013, in Cambodia, transportation of goods cost US\$10/100km/tonne. That's compared with US\$7/100km/tonne in Vietnam, and US\$5/100km/tonne in Thailand, according to the \href{https://www.worldbank.org/en/country/cambodia/publication/cambodian-agriculture-in-transition-opportunities-and-risks}{World Bank}. 

In terms of logistical performance Cambodia falls short of its neighbors as well. In 2018, Cambodia received a score of 2.58 from the \href{https://lpi.worldbank.org/international/scorecard/radar/254/C/KHM/2018}{World Bank}'s Logistics Performance Index, which takes into account factors like customs, infrastructure, international shipments, logistics quality and competence putting the country at a rank of 98 out of 160 countries. While Thailand received a score of 3.41 with rank 32 and Vietnam received a score of 3.71 with rank 39.

And the lack of information on market and market access, informal payments, intensive manual and administrative export procedures, and intense competition with imported products remain big challenges.

The Kingdom should help this sector by favoring taxation, technical and financial support to firms. Likewise, the public universities that provide agro-processing training, and stakeholders should contribute to the improvement of research and development (R\&D) with the enterprise to discover a new product, technical, and technological innovation.

In terms of taxation, the government of Cambodia must make supporting investment to the agro-processing industry a policy priority.

Both Thailand and Vietnam have given heavy policy support to their respective agro-processing industries. The Thai government, for example, created \href{http://foodinnopolis.or.th/en/home/}{Food Innopolis}, a global food innovation hub focusing on research, development, and innovation for the food industry. The platform offers multiple types of tax exemptions, including on research and development equipment, exclusive licenses to own land, permanent resident visa, and other benefits. Investors also can obtain additional funding from numerous government agencies.

Vietnam has made itself \href{https://www.seavestor.com/vietnams-food-processing-industry-a-promising-option-for-foreign-investors/}{highly attractive} for foreign and local investors in terms of food processing by offering \href{https://www.foodnavigator-asia.com/Article/2019/01/02/Prosperity-project-Vietnam-s-rapidly-growing-food-processing-industry-eyes-foreign-investment?utm_source=copyright&utm_medium=OnSite&utm_campaign=copyright}{preferential tax policies} for investors. \href{http://bizhub.vn/news/foreign-investors-pour-more-money-into-food-processing-and-drinks-industry_302792.html}{Vietnamese} processed agricultural products are currently being exported to as many as 200 countries.

The Cambodian government must also step up its support to R\&D for this sector. Thailand attributes much of its competitiveness in the food industry to its \href{http://www.boi.go.th/upload/content/Food%20industry_5aa7b40bd758b.pdf}{increased investment} in R\&D and technological innovation. Those innovations have facilitated an industry that maintains low physical costs and flexible manufacturing structures. But it is not easy to do so for Cambodia agro-processing SMEs that have limited cash flows; they cannot improve their workforce, capital, and technical progress.

In terms of improving financing, two institutions could have a critical role to play: the Agricultural and Rural Development Bank (ARDB) as well as the forthcoming SME Bank. In 2018, the ARDB provided US\$159.11 million of total loans to local enterprises. And, the SME Bank will soon operate in the hope of stimulating agro-processing and other SMEs with an initial capital of US\$100 million.

Cambodia can be doing more with its agricultural production than it is right now. But to capitalize on the potential, it will take willingness on the part of the government to put these policy pieces in place to attract and support investment in this industry. 

\textit{Nith Kosal is a young research fellow at Future Forum, an independent public policy think tank based in Phnom Penh. His research covers financial subsidy policy and agricultural price policy in Cambodia.
}



\end{document}