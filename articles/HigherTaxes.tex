\documentclass[10pt,a4paper]{letter}
\usepackage[latin1]{inputenc}
\usepackage{utopia}
\usepackage{amsmath}
\usepackage{amsfonts}
\usepackage{amssymb}
\usepackage{graphicx}
\usepackage{float}
\usepackage{url}
\usepackage{hyperref}
\usepackage[left=1in, right=1in, bottom=.9in, top=.9in]{geometry}
\renewcommand{\baselinestretch}{1.2}

\usepackage[table,xcdraw]{xcolor}
\usepackage[dvips]{xcolor}
\definecolorseries

\hypersetup{colorlinks,linkcolor={[cmyk]0,0.88502,1,0},citecolor={[cmyk]0,0.88502,1,0}, filecolor={violet}, urlcolor={[cmyk]0,0.88502,1,0}}  

\begin{document}
	
{\Large 
	\textbf{Pandemic Offers Chance to Consider Higher Taxes, not Donations, From Rich}
}

VOD, August 13, 2021, 17:50 ICT\\
By Nith Kosal

\textit{This moment should be used as an opportunity to re-evaluate Cambodia's policies around taxation --- specifically the income tax, and the rate at which the wealthiest Cambodians contribute to national coffers, writes Future Forum researcher Nith Kosal.}

Cambodia faces a significant challenge to find more sources of government revenue as it fights Covid-19. As a consequence of the pandemic, the government has taken a hit in revenues in several economic sectors: notably hotels, restaurants, tourism and transport.

But this difficult moment could also be used as a time of self-reflection. Covid has provided an opportunity for everyone, including the Cambodian government, to reform policies and measures to improve the public good in a range of areas: to prepare for future pandemics, to be more resilient to climate change, and to pursue sustainable economic development, including structural changes for the medium  and long term.

There are arguments to be made that this moment should be used as an opportunity to re-evaluate Cambodia's policies around taxation --- most specifically the income tax, and the rate at which the wealthiest Cambodians contribute to national coffers. Cambodia should move away from relying on foreign financing and the largesse of tycoons' donations, which raise the risk of corruption, toward instituting higher tax brackets in line with regional income-tax rates.


So far, the majority of the efforts from the government have revolved around economic recovery and \href{https://www.phnompenhpost.com/opinion/action-now-can-help-prevent-covid-19-pushing-more-cambodians-poverty}{social development}. The government has set up a comprehensive response and fiscal stimulus package to address the socioeconomic effects of the crisis. These actions have included cash transfers to poor and vulnerable households, wage subsidies to the unemployed, low-interest loans and tax relief for firms in hard-hit sectors, cash for agriculture programs, and support to small and medium-sized enterprises. All of these initiatives cost money: The government intervention was budgeted at \$823 million in 2020 and \$719 million in 2021.

At this time, we need more government expenditure to support economic recovery, social development and the health care system. But the country is also facing constraints regarding budgeting.

Due to gaps between revenue collection and financial requirements, authorities are taking out more loans from development partners. A total of \href{https://www.phnompenhpost.com/special-reports/cambodia-strives-keep-its-finances-together-despite-rising-impacts}{\$2.02 billion} in concessionary loans were signed in 2020, up 56 percent from \$1.3 billion in 2019, reaching 74 percent of Cambodia's debt ceiling.

From 2020 through 2023, the World Bank \href{https://www.worldbank.org/en/news/press-release/2021/06/16/cambodia-country-economic-update-june-2021-cambodia-s-economy-recovering-but-uncertainties-remain}{is projecting} that the Cambodian government will run at a deficit of about 3 to 4 percent of GDP. Most of this shortfall is set to be covered by foreign financing.

National income, meanwhile, has been consistently increasing over the past four years thanks to rising tax collection. But it is also slated to take a hit in the pandemic: Government revenues, which also include grants, mining permits, investments and private donations, fell 11 percent in the first two months of this year, according to the World Bank.

As the country emerges from the pandemic, Cambodia has the opportunity to improve its budgetary balance by further shifting revenues toward domestic sources, especially through reforming income taxes for the country's highest earners.

Currently the personal income tax rate is between \href{https://taxsummaries.pwc.com/cambodia/individual/taxes-on-personal-income}{0-20 percent} --- a rate of 0 percent for an individual who received an income below \$150 per month, 5 percent for an individual with a monthly income between \$150-\$312.5, 10 percent (\$312.5-\$2,125), 15 percent (\$2,125-\$3,125), and 20 percent for an income from \$3,125 and above.

A tax rate of 20 percent for high-income earners is far too low for individuals who bring in thousands of dollars each month.

Personal income tax rates are as high as 35 percent for top earners in our Asean neighbor countries of  \href{https://taxsummaries.pwc.com/vietnam/individual/taxes-on-personal-income}{Vietnam}, \href{https://www.rd.go.th/english/6045.html}{Thailand}, \href{https://santandertrade.com/en/portal/establish-overseas/malaysia/tax-system}{Malaysia}, and \href{https://taxsummaries.pwc.com/indonesia/individual/taxes-on-personal-income}{Indonesia}, showing that higher rates are achievable within this region. Theoretically, a progressive personal income tax system can contribute to generating higher national incomes and reducing inequality.

In countries like France, Germany, Japan, China and South Korea, personal tax rates for high-income individuals are around 45 percent or higher. And some countries tax their wealthiest at rates of 50 percent, like Belgium, Israel and Slovenia.

The government can improve and reform individual income tax policy. The personal income tax rate should range between 0-35 percent for us.

In lieu of a higher rate, historically the Cambodian government has relied on the generosity of large domestic firms and elites, who have often publicly offered donations. In response to issues like building collapses in Kep and Preah Sihanouk provinces, for example, many of the richest men contributed millions of dollars to the authorities. To \href{https://thediplomat.com/2020/12/cambodia-vaccine-push-offers-window-into-elite-networks/}{purchase Covid-19} vaccines, the government received more than \$30 million from these donors and the public.

But it isn't enough to rely on generosity alone. The government must institutionalize the revenue received from high-earning Cambodians. If wealthy patrons and businessmen can part with millions of dollars in donations in times of crisis, they should be able to withstand a 20-35 percent tax rate in stabler times.

A set, higher tax rate for wealthier Cambodians would not just allow the government to be more prepared to deal with turbulent moments, it's important to limit Cambodia?s reliance on private donations for other reasons as well --- namely, what extra benefits might be afforded to the businesses and individuals who make those donations.

According to the \href{https://www.transparency.org/en/cpi/2020/index/khm}{2020 Corruption Perceptions Index}, Cambodia has the worst score in the Asean region for corruption and was ranked 160 out of 180 countries overall.

A lot of corruption can involve tycoons. With the title ``oknha'' roughly equivalent to a nobleman, some have used the position to cover for \href{https://thediplomat.com/2021/05/cambodia-sets-up-working-group-to-scrutinize-granting-of-honorific-title/}{corrupt enterprises}, including illegal logging and sand mining and criminal activities like drug trafficking. Among these unethical businesses, the government has lost millions of dollars in tax collection due to underreporting on business progress and capital flows. The corruption of some government officials is the same. They open their hands to corrupt businessmen and exploit national budgets for personal gain.

In democratic societies, we want people to live equally under the rule of law, regardless of their wealth. Thus, taxing the rich at a rate commensurate with what they earn is an important strategy to reduce poverty and income disparity among the population.

To accomplish these changes to the tax system, the tax department should create a team to work with the Labor Ministry to study progressive tax systems in other countries and to examine their successes and challenges in the realm of policy implementation. And then, they should identify high-income individuals and classify them at a rate of 20-35 percent. 

Now the time has come for the government, civil society organizations, development partners and citizens to concentrate on working together to develop more internal sources of revenue for the nation rather than external borrowing. An effective solution would be to look to the country's highest earners and make sure they are paying their fair share.


\textit{Nith Kosal is a junior research fellow at Future Forum. His research interests include applied economics, macroeconomics, economic development and international economics.
}





\end{document}